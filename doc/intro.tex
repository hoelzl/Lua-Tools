\chapter*{Introduction}
\addcontentsline{toc}{chapter}{Introduction}

The \emph{Pseudo-Operational Ensemble Modelling language} (POEM) is developed by the ASCENS project\footnote{See \cite{ASCENS} for more on the project.} as a tool to model the behavior of \emph{software ensembles}. As defined in \cite{AscensTR7}, ``[e]nsembles are software-intensive systems with massive numbers of nodes or complex interactions between nodes, operating in open and non-deterministic environments in which they have to interact with humans or other software-intensive systems in elaborate ways.'' In order to accurately describe ensembles, POEM chooses a two-layered approach by using ``an action programming language based on classical first-order logic as the foundation, and on top of this basic layer support for probabilistic reasoning and soft-constraint-based optimization.''\footnote{Cf. \cite{AscensTR7} (which is cited here) or \cite{AscensJD21} for more on the POEM language.} Thus, the POEM language provides a very fancy set of high-level tools to model the behavior of a given ensemble.

However, one of the use case scenarios, which the ASCENS project seeks to apply the POEM language to, consists of a software ensemble made up by a group of small robots, which need to coordinate their actions to achieve a common goal and, unluckily, whose hardware specifications are rather limited. To provide these robots with an adequately resource-efficient version of the POEM language interpreter in the future, it is necessary, first of all, to implement the basic functionalities the POEM language builds upon in a language suitable to run on embedded systems (like our robots), which shall be the goal of this paper.

All programs presented in this paper are written in Lua, which is an open-source scripting language, whose flexibility and dynamic natures allows for fast coding but, more importantly, whose compatibility with ANSI C and small interpreter footprint makes it a premier choice for embedded systems.\footnote{When in-depth knowledge of Lua mechanisms is required to understand a presented piece of software, it will be explained in the respective section. However, a basic understanding of Lua's design principles is probably necessary to fully understand all listings in this paper. Cf.  \cite{Ierusalimschy2006} for an extensive and well-written introduction to Lua.} As Lua's standard library is relatively small, too, one of the downsides of using Lua is the necessity to include an implementation of relatively basic features to be used in the project.

Especially chapter \ref{chap:tools} describes three useful modules, which provide a way to serialize Lua data (\ref{sec:tools:serialize}), perform some slightly advanced operations on Lua's ubiquitous table data type (\ref{sec:tools:typetools}) and execute a function in a protected environment, thus enabling the straight-forward simulation of non-deterministic behavior, which is an integral part of the POEM language, in a Lua program (\ref{sec:tools:nd}).

The modules presented in chapter \ref{chap:messaging} provide messaging capabilities to the programmer, using the same interface to send messages to a shared message queue, possibly inside a shared memory segment (\ref{sec:messaging:shared}), or across the network by building a TCP connection (\ref{sec:messaging:distributed}). Both of these forms of message transmission can occur between the component of an ensemble.

Chapter \ref{chap:oo} mainly describes the module \texttt{oo}, which offers a flexible yet efficient framework for object-oriented programming. It unifies different approaches at implementing objects in Lua to a single interface (\ref{sec:oo:usage}), while enabling the user to dynamically switch between multiple of these approaches that are provided within the module (\ref{sec:oo:implementation}).

In chapter \ref{chap:evolution}, this OO framework is used to implement an abstract environment object to collect all data needed for the specification of a constraint problem (\ref{sec:evolution:constraints}) and providing a few (numeric) search domains (\ref{sec:evolution:domains}). To solve a constraint satisfaction or optimization problem defined in such a manner, this paper presents and discusses both a constraint solver using \emph{simulated annealing} (\ref{sec:evolution:sa}) and one using \emph{evolutionary algorithms} (\ref{sec:evolution:evolution}). Complementing each other in regard to suitable problems, these modules provide a viable realization of the soft-constraint solver to come with the POEM language.

This paper's appendices include a few example scripts in order to showcase how to use the described modules for a productive software (A, B, C), as well as an overview over all functions provided by the modules introduced in the previous paragraphs (D).

Note that since the only dependencies between the chapters come from latter modules using usually trivially explained functionality\footnote{The main features of earlier modules used in later ones are (1) data serilaization, (2) simple table operations not included in Lua's standard library, (3) non-deterministic branch and (4) object-oriented programming. The features (1-3) are covered in chapter \ref{chap:tools}, feature (4) is the main topic of chapter \ref{chap:oo}.} provided by previous modules, an experienced programmer should be able to read each chapter individually. After all, the software this paper is most aptly described as a set of tools aiding the implementation and usage of a POEM-like language.

\thispagestyle{plain}








