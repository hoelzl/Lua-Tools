\chapter{Modules \texttt{serialize}, \texttt{typetools} and \texttt{nd}: Tools}
\label{chap:serialize}
\label{chap:typetools}
\label{chap:nd}

The three modules discussed in this chapter were mainly created in order to provide functions that are needed by other (more complex) modules but are independent on their respective main purpose, so that it was a natural step to decouple parts of the functionality.

Understanding these auxiliary modules is not necessary in order to use the more high-level modules that build upon them: In most cases, the functionality provided by these modules might as well fit into the standard library and the functions provided by these modules work like an experienced programmer would expect them to. If that's not the case, they will be discussed shortly in the chapter of the modules they are used in.

\section{Module \texttt{serialize}: Serializing arbitrary data structures}
\label{sec:tools:serialize}

\subsection{The problem of data serialization}
\label{sec:tools:serialize:problem}

The module \texttt{serialize} is mainly used in the \texttt{distributed} script (see chapter \ref{chap:distributed}) to transmit Lua data structures over the network but has also proven extremely useful for many different I/O tasks and debugging complex data structures. It allows the programmer to convert data structures into a literal that can be parsed by a Lua interpreter to retain an equivalent data structure. The module \texttt{serialize} is able to serialize data of the Lua types \texttt{boolean}, \texttt{number}, \texttt{string} and \texttt{table}.\footnote{Note that as Lua provides not enough reflection possibilities on functions, \texttt{serialize} cannot work on them. Thus, \texttt{serialize} is also unable to serialize objects like the ones defined by the module \texttt{oo} (see chapter \ref{chap:oo}).} Serialization is trivial for the first three of these data types in Lua,\footnote{Note that Lua's strings are primitives and non-referential types (contrary to strings in Java, fore example).} section \ref{sec:tools:algorithm} focuses on table serialization.

Once serialized, a data structure can be read using Lua's \texttt{loadstring} function, which takes a string (of hopefully Lua code) and returns a \emph{function} that executes the given code when called. This behavior brings the advantage that parsing and executing the code can be done separately, but it also allows parameters to be passed into the given code. And like any other function in Lua, the code given introduces its own local namespace and allows for lexical binding. Also, if we want to extract a data structure from the given code (without writing it in a variable and thus affecting the surrounding program), we need to use a \texttt{return} statement.

\begin{lstlisting}[language=lua, caption={Using Lua's \texttt{loadstring} function}, label=lst:loadstring, name=lst:loadstring]
x = 4
f = loadstring("return ... + x") -- "..." denotes the parameter
print(type(f))                   -- prints "function"
print(f(5))                      -- prints 9
\end{lstlisting}

\subsection{Usage}
\begin{comment}
TODO: Adjust actual function serialize.command to accept lists of parameters
\end{comment}

The function \texttt{serialize.data} returns the serialization of the given parameter as a string. \texttt{serialize.command} allows to serialize a statement by passing it both a function name and a list of parameters. The generated code will then return the result of calling the given function with the given arguments. This is especially useful when transmitting programs or program parts between systems.\footnote{This is done by the script \texttt{distributed} in section \ref{sec:messaging:distributed}.}

\begin{lstlisting}[language=lua, caption={Using \texttt{serialize.data} and \texttt{serialize.command}}, label=lst:serializedatacommand, name=lst:serializedatacommand]
require "serialize"
x = 4
sx = serialize.data(x)
print(sx)                            -- prints "return (4)"
print(loadstring(sx)())              -- prints 4
sp = serialize.command("print", x)
print(sp)                            -- prints "return print(4)"
loadstring(sp)()                     -- prints 4
\end{lstlisting}

Since recursive tables cannot be written as table literals in Lua, \texttt{serialize} and all its functions use a slightly different format for table serialization: The return statement is prefixed by a prologue of local variable declarations. Although this makes the serialized code a bit lengthy, it ensures that no declarations in the current program are changed upon loading the serialized data structure.

\begin{lstlisting}[language=lua, caption={Serializing a recursive table}, label=lst:serializeprint, name=lst:serializeprint]
require "serialize"
t = {x = 4, y = 5}
t.z = t
serialize.print(t)         -- prints 'local x1={}; x1["y"]=5; x1["x"]=4; x1["z"]=x1; return (x1)'
\end{lstlisting}

Note that \texttt{serialize.print} is a shortcut that serializes all given parameters and then prints them using the global \texttt{print}.

\subsection{Algorithm}
\label{sec:tools:algorithm}

Both \texttt{serialize.data} and \texttt{serialize.command} use the internal function \texttt{serialize} (of the module \texttt{serialize}). This section will show briefly how it works. Users of the module should not be concerned about it and this section is merely meant as an in-depth explanation for programmers that want to extend \texttt{serialize}.

The algorithm used in \texttt{serialize} closely resembles the serialization algorithm given in section 12.2 of \cite{Ierusalimschy2006}, though it does not require a name for the initial data structure because it serializes to Lua chunks (using the \texttt{return} key word as discussed in section \ref{sec:tools:serialize:problem}).\footnote{The version presented in \cite{Ierusalimschy2006} is targeted at persistance and thus aims to restore the global namespace from the serialized data. The module \texttt{serialize} opts for message passing and thus aims to keep the global namespace unaffected.} To keep track of the current progress of serialization, the function \texttt{serialize} needs three accumulator parameters that are passed on to recursive calls of the function, which are necessary iff it is passed a table. When serializing a data structure that is not part of larger one, these parameters can be left out\footnote{It is a common Lua idiom to include them in one function which uses the \texttt{or} operator to set these parameters to their default values if necessary, as it is done in listing \ref{lst:serialize}.}, otherwise they carry a lookup table of all currently serialized data structures (\texttt{saved}), which allows the algorithm to fill in their already-assigned name and avoid running into an infinite loop on self-containing tables, the code for all current declarations (\texttt{prologue}) and the \texttt{index} that is used to generate new, unique variable names.

\begin{lstlisting}[language=lua, caption={The private function \texttt{serialize} of the module \texttt{serialize}}, label=lst:serialize, name=lst:serialize]
local function serialize(expr, saved, prologue, index)
	saved = saved or {}
	prologue = prologue or ""
	index = index or 1
	if type(expr) == "number" or type(expr) == "boolean" then
		return tostring(expr), saved, prologue, index
	elseif type(expr) == "string" then
		return string.format("%q", expr), saved, prologue, index
	elseif type(expr) == "table" then
		if not saved[expr] then
			saved[expr] = "x"..tostring(index)
			prologue = prologue..string.format("local %s={}; ", saved[expr])
			index = index + 1
			for key,val in pairs(expr) do
				kexpr, saved, prologue, index = serialize(key, saved, prologue, index)
				vexpr, saved, prologue, index = serialize(val, saved, prologue, index)
				prologue = prologue..string.format("%s[%s]=%s; ", saved[expr], kexpr, vexpr)
			end
		end
		return saved[expr], saved, prologue, index
	else
		error("cannot serialize "..type(expr))
	end
end
\end{lstlisting}

Exposing these accumulators to user of the function \texttt{serialize} allows for precise combination of multiple serialization processes, as done by \texttt{serialize.command}, for example.


\section{Module \texttt{typetools}: Table manipulation}

\subsection{Lua's metatables}
\label{sec:metatables}

Lua's only native complex data structure are tables, which can be used as an adequate representation of most other complex data structures.\footnote{Chapter 11 of \cite{Ierusalimschy2006} provides many example implementations of well-known data structures in native Lua using tables. Chapter \ref{chap:oo} of this paper expains mutliple implementation of objects in Lua, which all involve tables at some point.} However, tables can also be used to replace basic data structures like Lua's other native data types to a certain extent. A table's behavior can be fine-tuned using the \emph{metatable} mechanism of Lua: Each table can be assigned exactly one metatable, which is a table used to contain special information for the Lua interpreter about how to handle certain situations. These situations are typically type errors that the interpreter can avoid using the functions provided by the metatable. For example, it is usually not possible to add two tables using Lua's \texttt{+} operator. However, if those tables' metatables\footnote{If both addends are tables, Lua refers to the first one's metatable first, then to the second one's. So technically, only one of their metatables needs to define an entry \texttt{"\_\_add"}} define an entry with the key \texttt{"\_\_add"}, this entry's value (hopefully a function) will be called instead of raising an error. Metatable entries that are expected to contain a function (usually in order to simulate the behavior of a built-in operator) are also called \emph{metamethods}.

\begin{lstlisting}[language=lua, caption={Defining vector addition using the metamethod \texttt{"\_\_add"}}, label=lst:vecadd, name=lst:vecadd]
vec1 = {1, 2, 3}
vec2 = {7, 8, 9}
-- vec = vec1 + vec2 -- throws error: "attempt to perform arithmetic on global 'vec1' (a table value)"
metavec = {
	__add = function (a, b)
		local result = {}
		for i,_ in ipairs(a) do
			result[i] = a[i] + b[i]
		end
		return result
	end
}
setmetatable(vec1, metavec)
setmetatable(vec2, metavec)
vec = vec1 + vec2 -- work as expected
print(vec[3]) -- prints 12
\end{lstlisting}

Lua provides a number of metamethods in order to enable (a certain extent of) \emph{type virtualiziation}, i.e. simulating other Lua types using tables and metatables.\footnote{For an extensive list of possible virtualizations, see the article \emph{LuaVirtualization} in \cite{LuaUsersWiki}. For an extensive list of all available metamethods, see the article \emph{MetatableEvents}.} However, the module \texttt{typetools} mainly uses the metatable mechanism to work on tables, as using this method, a table can also simulate to contain the entries of another table: The metamethod \texttt{"\_\_index"} is called every time the program tries to access an \emph{undefined} entry in the table the metamethod is defined for. The function \texttt{typetools.proxy} utilizes this "hook" to provide a fast method for (simulating) copying a table: As shown in listing \ref{lst:proxy}, it returns an empty table whose metatable redirects every access to a deep copy of the given target's respective entry. When accessing the whole table, \texttt{typetools.proxy} is just an elaborate version of a common deepcopy algorithm, but when it is clear that only small parts of the original table are actually accessed destructively, \texttt{typetools.proxy} can save memory and computation type.\footnote{An example of exactly that situation is shown in section \ref{sec:whatif}.}

\begin{lstlisting}[language=lua, caption={The function \texttt{typetools.proxy}}, label=lst:proxy, name=lst:proxy]
function proxy(target)
	if type(target) == "table" then
		local proxytable = {}
		local proxymetatable = {
			__index = function(table, key)
				table[key] = deepcopy(target[key])
				return table[key]
			end
		}
		setmetatable(proxytable, proxymetatable)
		return proxytable
	else
		return target --only non-destructive operations on non-tables anyway
	end
end
\end{lstlisting}

Note that \texttt{typetools.proxy} is not a transparent replacement for a "true" copy algorithm and cannot be in Lua 5.1 as table virtualization is not fully possible.

\subsection{More table hooks}

The main problem with table virtualization in the current Lua version\footnote{This paper is based on Lua 5.1.5.} is the lack of a hook to catch Lua's built-in iteration function \texttt{pairs} and \texttt{ipairs}. The module \texttt{typetools} provides a replacement for Lua's standard function \texttt{pairs} that explicitly calls the (conventional) metamethod \texttt{"\_\_pairs"} before performing the same task as the built-in \texttt{pairs} function.\footnote{The module \texttt{typetools} follows a suggestion in the article \emph{GeneralizedPairsAndIpairs} of \ref{LuaUsersWiki2006} that when a \texttt{"\_\_pairs"} metamethod is established, it should be used to differentiate between an unordered iteration (\texttt{pairs}) and an ordered one (\texttt{ipairs}) instead of having the user of the table make that decision.} Thus, \texttt{typetools.pairs} can overcome the table virtualization issue, but in order to ensure compatibility, it needs to be used in \emph{every} table iteration in a program, which is impractical in most cases.

The function \texttt{typetools.proxy} shown in listing \ref{lst:proxy} above calls (within the same namespace) the function \texttt{typetools.deepcopy}, which basically works as expected and thus fills a hole in Lua's standard library. The module \texttt{typetools} also provides its pendant, \texttt{typetools.shallowcopy}. However, \texttt{typetools.deepcopy} refers to its argument's metatable, too: Before performing an expensive deep copy algorithm, the function checks for the metamethod \texttt{"\_\_copy"} and (if it is present) returns its result. Note that the metamethod \texttt{"\_\_copy"} is not used or defined by the Lua interpreter. Just by using a certain entry of the metatable, the programmer can implicitly introduce new metamethods as it done by \texttt{typetools}. The deep copy algorithm can thus provide a special form of \emph{polymorphism} as it will work differently on different tables.\footnote{The objects defined by the module \texttt{oo} use this behavior of \texttt{typetools.deepcopy} to have it call their own copy method \texttt{clone}. See section \ref{sec:oo:copying} for more on the method \texttt{oo.object:clone}.}


\section{Module \texttt{nd}: Non-deterministic program flow}

\subsection{State in a Lua program}

The main purpose of the module \texttt{nd} is to provide a non-deterministic choice operator (\texttt{nd.orelse}) to allow the user to split the program flow into independent branches, which is not easily accomplished using the standard tools of a script language like Lua. The basis of non-deterministic execution is the function \texttt{nd.whatif}, which is very useful for a variety of search problems.\footnote{Like the constraint solving performed by the module \texttt{constraints} (see chapter \ref{chap:constraints}).} It allows to shield the program's state from changes by the function passed as its first argument and returns this function's return value as if it was executed under normal circumstances.\footnote{\texttt{nd.whatif} also returns a few tables with variable bindings, which will be discussed later in this section.} See listing \ref{lst:nd} for an example.

\begin{lstlisting}[language=lua, caption={Preventing global changes with \texttt{nd.whatif}}, label=lst:nd, name=lst:nd]
require "nd"
x = 5
more = function ()
	x = x + 1
	return x
end
print((nd.whatif(more))) -- prints 6
print(x) -- prints 5
\end{lstlisting}

This behavior is achieved in Lua by saving the complete global state before executing the given function. (We will call the first argument to \texttt{nd.whatif} \emph{possibility} from now on.) As Lua supports lexical binding inside lambdas (i.e., \emph{closures}), the possibility can access all variables of its enclosing namespaces and thus cause side effects in any part of the software that shares that variables. From a Lua perspective, the relevant variables can be split into two categories: Global variables are defined by the possibility's \emph{function environment}, which is usually set to the current global environment\footnote{For readers unfamiliar with Lua: Lua saves the global environment as a table with the variable names as keys. It can be accessed directly through the global variable \texttt{\_G} (i.e. \texttt{\_G.\_G == \_G}). Thus, the global environment \emph{is} just a common data structure that can easily be passed to other functions.} of the function's definition. Shared local variables are defined through standard lexical binding and are also called \emph{upvalues} of a function. Both of these cases need to be handled differently in Lua.

\subsection{Function introspection}
This (sub)section describes the means by which the function environment and upvalues of a function can be accessed in Lua and has a more technical nature. If the reader is only interested in using the module \texttt{nd} through its interface, one should feel free to skip to section \ref{sec:whatif}.

Lua supports accessing and changing a function's environment through the function pair \texttt{getfenv}/\texttt{setfenv}. After retrieving the possibility's function environment, \texttt{nd.whatif} applies \texttt{typetools.proxy} (see section \ref{sec:metatables}) to protect it from changes without generating a possibly unnecessary deepcopy and uses the result as the possibility's new function environment. When the possibility returns, the original function environment is restored. Potential changes made by the possibility to its global environment only affect the current global environment.

\begin{lstlisting}[language=lua, caption={Changing a function's environment.}, label=lst:fenv, name=lst:fenv]
x = 5
more = function ()
	x = x + 1
	return x
end
fenv = getfenv(more)
print(fenv.x) -- prints 5
setfenv(more, {x = 7})
print(more()) -- prints 8
\end{lstlisting}

Accessing upvalues is a bit more tricky and requires the use of Lua's standard library \texttt{debug} since the required functionality reaches into metaprogramming territory. Since upvalues are local variables, whose bindings can potentially be overshadowed, they aren't accessed by their name but by their numerical index. The function \texttt{debug.upvalue} receives a function and an integer as arguments and returns the respective local name and upvalue, \texttt{debug.setupvalue} works likewise (receiving the new value as an additional parameter).

\begin{lstlisting}[language=lua, caption={Setting a function's upvalue "from the outside".}, label=lst:upvalue, name=lst:upvalue]
local x = 5
more = function ()
	x = x + 1
	local x = x * 2
	return x
end
print(more(), x) -- prints 12, 6
debug.setupvalue(more, 1, 20)
print(more(), x) -- prints 42, 21
\end{lstlisting}

As listing \ref{lst:upvalue} shows, the changes to a function's upvalues do not strictly concern the function but every part of the program that shares the respective lexical binding. Again, \texttt{nd.whatif} uses this mechanism to save all\footnote{The total number of upvalues of a function can be retrieved through the call \texttt{debug.getinfo(call, "u").nups}. See section 23.1 of \cite{Ierusalimschy2006} for more information.} upvalues before the execution of the possibility and write back these original values afterwards. However, \cite{Ierusalimschy2006} explicitly notes that using the \texttt{debug} library's functionality needed to access upvalues is "not efficient". Thus, it is recommended to turn off the local namespace preservation of \texttt{nd.whatif} wherever reasonable, as it is shown in the following section.

\subsection{Asking \texttt{whatif}}
\label{sec:whatif}

To allow for a fine-tuned resource management and great flexibility, \texttt{nd.whatif} supports elaborate call patterns for many situations. In general, \texttt{nd.whatif} accepts two environments (as its second and third argument), whose bindings will be applied to the global or local namespace respectively before executing the possibility.

\begin{comment}
TODO: update nd.whatif to include an option (_true_ or nil?) to ERADICATE said namespace and execute in plain environment.
\end{comment}

 \begin{table}[h]
 \caption{Call patterns of \texttt{nd.whatif}}
 \begin{tabular}{l|p{300pt}}
argument  & behavior\\
\hline
\texttt{false} & don't prevent changes in the respective namespace\\
\texttt{true}, \texttt{\{\}} or \texttt{nil} & prevent changes in the respective namespace \\
\texttt{\{...\}} & apply given bindings before execution and prevent any changes in the original namespace\\
 \end{tabular}
 \label{tab:whatif}
 \end{table}

\begin{lstlisting}[language=lua, caption={Example for the call patterns of \texttt{nd.whatif} (see table \ref{tab:whatif})}, label=lst:whatif, name=lst:whatif]
require "nd"
x = 5
local y = 1
more = function ()
	x = x + y
	return x
end
print((nd.whatif(more, true, false))) -- save computation time by not saving the upvalues (also print 6)
nd.whatif(more, false, true)          -- allow access to global namespace
print(x)                              -- print 6
print((nd.whatif(more, {x=20}, {y=22}))) -- print 42
nd.whatif(more, false, false)         -- elaborately call more()
\end{lstlisting}

Building on \texttt{nd.whatif}, one can implement arbitrary combinations of non-deterministic program execution. A premier example of this is the function \texttt{nd.orelse}, which takes an arbitrary number of function as its arguments and executes the first one that does not return \texttt{nil} (and returns that function's result). It does so by trying all given functions using \texttt{nd.whatif} and then incorporating the changes done by the first successful function to its global variables and upvalues into the further program flow.


\begin{lstlisting}[language=lua, caption={Using \texttt{nd.orelse}.}, label=lst:orelse, name=lst:orelse]
require "nd"
x = 5
test1 = function ()
	x = 6
	return nil
end
test2 = function ()
	x = 7
	return true
end
test3 = function ()
	x = 8
	return true
end
print(nd.orelse(test1, test2, test3)) -- prints true
print(x)                              -- prints 7
\end{lstlisting}

\begin{comment}
TODO: Include reference to the usage of nd.whatif in distributed
\end{comment}
