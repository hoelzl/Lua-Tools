\begin{appendices}

\chapter{Manual messaging}

\section{Module \texttt{onsite}: Trivial implementation of the request interface}
\label{sec:app:onsite}

There is not much to say about the module \texttt{onsite}. It provides a trivial implementation of the request interface needed by the modules \texttt{shared} and \texttt{distributed} while printing out notes on what it does. It is part of this software because without any implementation of this function, \texttt{shared} and \texttt{distributed} could not even be tested.

It is also a good showcase of the modularization technique used in all of the modules: Even though Lua provides the function \texttt{module} that creates a a separate namespace for the current module, most modules need access to outside variables from the standard library or likewise required modules. For this scenario, Lua allows to pass the constant \texttt{package.seeall} as a second parameter to \texttt{module}. This results, however, in some strange phenomena that should be avoided in good software design.\footnote{See the article \emph{LuaModuleFunctionCritiqued} in \cite{LuaUsersWiki}.} Instead, it is possible to create local references to required global resources manually, thereby also creating a kind of dependency list of the module. Using this approach, the module \texttt{onsite} looks like this:

\begin{lstlisting}[language=lua, caption={The module \texttt{onsite}}, label=lst:onsite, name=lst:onsite]
local tostring = tostring
local print = print
local serialize = require "serialize"
module(...)

function get(query)
	local result = "response to "..tostring(query)
	print("local get of ", serialize.data(query), " returns ", result)
	return result
end
function qry(query)
	local result = "response to "..tostring(query)
	print("local qry of ", serialize.data(query), " returns ", result)
	return result
end
function put(fact)
	print("local put of ", (serialize.data(fact)))
	return fact
end
\end{lstlisting}

\section{Module \texttt{shell}: Command line interface}
\label{shell}

The module \texttt{shell} was developed as part of a test application of the messaging functionalities provided by the modules \texttt{shared} and \texttt{distributed} (see chapter \ref{chap:messaging}). It implements a simple command line interface that allows the user to send and answer messages by entering the respective commands into the prompt. These commands include the three basic message types supported by the modules \texttt{shared} in \texttt{distributed} (i.e. \texttt{get}, \texttt{put} and \texttt{qry}) as well as the command \texttt{answer} to trigger the answering process for the earliest pending message.\footnote{For more information on the message protocol, refer to chapter \ref{chap:messaging}.} The shell can be exited by entering the command \texttt{quit}.

However, the module \texttt{shell} is not meant for direct execution by the user, as it cannot initiate the messaging mechanisms itself. It is easy to built a small Lua script to do that, though. The following listings \ref{lst:shell:shared} and \ref{lst:shell:distributed} show simple scripts which initiate a messaging module (\texttt{shared} or \texttt{distributed} respectively) and run the shell to pass on control to the user.

\begin{lstlisting}[language=lua, caption={A script starting \texttt{shared} messaging and running a \texttt{shell}}, label=lst:shell:shared, name=lst:shell:shared]
remote = require "shared"
require "onsite"
require "shell"

remote.init(onsite, "localhost")
print(shell.help())
print([[
  The prefixed number indicates the current coroutine. New ones are produced on the fly
  when a coroutine blocks. Use the stop command to switch coroutines.]])
local calls = {}
local current = 1
local max = 1
while true do
	calls[current] = calls[current] or coroutine.create(function() shell.run("localhost", tostring(current)) end)
	if coroutine.status(calls[current]) == "dead" then
		print("An unexpected error occured.")
		os.exit() --calls should never die anyway
	end
	if coroutine.status(calls[current]) == "suspended" then
		local success, unblock = coroutine.resume(calls[current])
		if not success then print(unblock); os.exit() end --makes debugging easier
		if not unblock then max = max + 1 end
	end
	current = current < max and current + 1 or 1
end
remote.term()
\end{lstlisting}

\begin{lstlisting}[language=lua, caption={A script starting \texttt{distributed} messaging and running a \texttt{shell}}, label=lst:shell:distributed, name=lst:shell:distributed]
remote = require "distributed"
require "onsite"
require "shell"

remote.init(onsite, arg[1], arg[2])
print(shell.help())
shell.run(arg[3] or "localhost")
remote.term()
\end{lstlisting}

Listing \ref{lst:shell:shared} is a bit more complex because it contains the coroutine infrastructure: To simulate multiple parallel messaging components within one Lua script, the script works with coroutines. Whenever a message awaits an answer, it passes on the control of the processor to the next coroutine.\footnote{New coroutines are created on the fly as soon as a coroutine blocks.} The current coroutine is indicated by the number at the beginning of the prompt. An example log of using this script could look like this:

\begin{lstlisting}[language=tex, caption={A log from using the script of listing \ref{lst:shell:shared}}, label=lst:shell:sharedlog, name=lst:shell:sharedlog]
  Type cmds for a list of available commands. Type help <command> for help.
  The prefixed number indicates the current coroutine. New ones are produced on the fly
  when a coroutine blocks. Use the stop command to switch coroutines.
1> get 3
2> a
local get of 	return (3)	 returns 	response to 3
2: true
2> stop
1: response to 3
1> q
\end{lstlisting}

Note that everything behind a ``\texttt{>}'' is user input. Here, the user first sends a \texttt{get} message with the parameter 3. It blocks the current coroutine 1 so the shell switches to coroutine 2 for the next prompt. The command \texttt{a} is a shortcut for \texttt{answer}, so the current coroutine receives a pending message and generates its response (using the module \texttt{onsite} to execute the local requests, as described in the previous section \ref{sec:app:onsite}). Line 6, starting with ``\texttt{local ...}'', is printed by the module \texttt{onsite}. The \texttt{answer} command returns \texttt{true} and coroutine 2 can prompt again. However, if the user wants to see the answer to the message sent in coroutine 1, it is necessary to switch back to that coroutine. The special command \texttt{stop} (which only has a meaning when working with coroutines) iterates between coroutines by blocking the current one. Thus, after its execution, the stop command passes on the control to coroutine 1, which immediately continues by printing the answer to the message ``\texttt{get 3}''. Finally, the command \texttt{q} is a shortcut for \texttt{quit} and exits the application.

Listing \ref{lst:shell:distributed} looks much simpler, but needs to be run twice\footnote{Since this \emph{is} about network communication, it does not necessarily need to be run on the same machine. However, the scripts presented here are configured to send all messages to ``localhost'', so they are meant to be run on the same computer.} in order two communicate over the network. After initiating the messaging module, it prints the shell's help message and calls \texttt{shell.run} to start the shell. A conversation between two users running this script could look like this:

\begin{lstlisting}[language=tex, caption={A log from using the script of listing \ref{lst:shell:distributed}}, label=lst:shell:distributedlog1, name=lst:shell:disributedlog1]
  Type cmds for a list of available commands. Type help <command> for help.
> get 42
: response to 42
> q
\end{lstlisting}

Note, however, that the program blocked execution after entering the command \texttt{get 42}. It only resumed because it received the answer it expected from another instance of the script it was configured to work with, to be exact, the following one:

\begin{lstlisting}[language=tex, caption={Another log from using the script of listing \ref{lst:shell:distributed}}, label=lst:shell:distributedlog2, name=lst:shell:disributedlog2]
  Type cmds for a list of available commands. Type help <command> for help.
> a
local get of 	return (42)	 returns 	response to 42
: true
> q
loca
\end{lstlisting}

The module \texttt{shell} itself mainly consists of a library of all available commands saved in the table \texttt{commands} and the function of the following listing \ref{lst:shell:run}. Note that each command itself is saved as a table containing two processing steps\footnote{The step called \texttt{"preprocess"} is called on the argument in string form and expected to return any Lua data type, while the step \texttt{"process"} is meant to execute the requested functionality.} as well as a help string.

\begin{lstlisting}[language=lua, caption={The function \texttt{shell.run}}, label=lst:shell:run, name=lst:shell:run]
function run(stdserver, prefix)
	prefix = prefix or ""
	while true do
		io.write(prefix, "> ")
		local command = io.read("*line")
		local recognized = false
		for name,steps in pairs(commands) do
			if string.match(command, "^"..name.."$") or string.match(command, "^"..name.."%s") then
				recognized = true
				local param = steps.preprocess(string.gsub(command, "^"..name.."%s+(.*)$", "%1"))
				local result = steps.process(stdserver, param)
				io.write(prefix, ": ", tostring(result), "\n")
				--break
			end
		end
		if not recognized then
			io.write("  unrecognized command\n")
		end
	end
end
\end{lstlisting}

The function \texttt{shell.run} prints a prompt using Lua's \texttt{io} module and then iterates over all commands\footnote{Obviously, this script only works when there are only a few available commands.} and checks for a match. It can then proceed to extract possible parameters and execute the requested functionality to print out the result.

Using this module, one can test the full range of messages supported by the modules \texttt{shared} and \texttt{distributed}.

\chapter{Script \texttt{queens}: A common optimization example}
\label{chap:app:queens}

\section{The problem's domain}
\label{sec:app:queens:domain}

The \emph{8 queens problem} is a widely known optimization problem. The task is to find a constellation of 8 queens on a standard $8 \times 8$ chess board so that none of the queens can check any other.\footnote{Every queen is able to check any other piece that she can reach by traveling horizontally, vertically or diagonally. Cf. section 3.2.1 of \cite{RussellNorvig2003} for an illustrated and more detailed explanation.}

From this problem description, it immediately follows that no two queens can be positioned on the same line or row of the chess board. Thus, the search space can be reduced (drastically) by including permutations of queens' columns, which are trivially assigned to the rows. For example, a (not so good) solution candidate may be \texttt{\{a=1, b=2, c=3, d=4, e=5, f=6, e=7, g=8\}}, a possible solution to the queens problem would be \texttt{\{a=6, b=2, c=7, d=1, e=3, f=5, e=3, g=8\}}. Using this representation, it is only necessary for the constraint environment to count how many queens check each other diagonally, as horizontal and vertical checks are ruled out by the data structure already.

This data structure, however, is not included in the \texttt{domains} module and thus, the user has to provide it. This is done in the first half of the script \texttt{queens} (see listing \ref{lst:queens1} below). Before doing so, the script sets the (for this software package) conventional \texttt{RNG} variable to allow the user to hot-plug another random number generator when calling the script and includes relevant modules. The global \texttt{topology} provides a mapping between the column names and an index, allowing later parts of the program to quickly compute the distance between two columns.

\begin{lstlisting}[language=lua, caption={The script \texttt{queens} (part 1)}, label=lst:queens1, name=lst:queens]
RNG = RNG or math.random

require "oo"
require "typetools"
require "domains"
require "constraints"
require "serialize"
require "evolution"
require "sa"

topology = {a=1, b=2, c=3, d=4, e=5, f=6, g=7, h=8}

queensdomain = oo.object:intend{
    columns = {"a", "b", "c", "d", "e", "f", "g", "h"},
    start = oo.public (function (self)
        local placement = {}
        local pos = {}
        for i,c in ipairs(self.columns) do
            local r = RNG(1, 8)
            while pos[r] do
                r = RNG(1, 8)
            end
            placement[c] = r
            pos[r] = true
        end
        return placement
    end),
    step = oo.public (function (self, origin)
        local x, y = RNG(1, 8), RNG(1, 8)
        local placement = typetools.deepcopy(origin)
        placement[self.columns[x]], placement[self.columns[y]] = placement[self.columns[y]], placement[self.columns[x]]
        return placement
    end),
    combine = oo.public (function (self, a, b)
        local cut = RNG(1, 7)
        local placement = {}
        local pos = {}
        for i=1,cut do
            placement[self.columns[i]] = a[self.columns[i]]
            pos[placement[self.columns[i]]] = true
        end
        local bi = 1
        for i=cut+1,8 do
            while pos[b[self.columns[bi]]] do
                bi = bi + 1
            end
            placement[self.columns[i]] = b[self.columns[bi]]
            pos[placement[self.columns[i]]] = true
        end
        return placement
    end)
}
\end{lstlisting}

The object \texttt{queensdomain} provides the domains for the queen constellation.\footnote{The way mutation (i.e. the domain's \texttt{step} method) and recombination (i.e. the domain's \texttt{combine} method) are implemented follows both section 2.4.1 of \cite{EibenSmith2007} and section 4.3.4 of \cite{RussellNorvig2003}, which use the same general approach.} At \texttt{start}, it positions a queen onto each column, remembering the rows of previous queens in the local table \texttt{pos}.\footnote{This table saves the position as a key and \texttt{true} as the value, which is a common implementation of a set in Lua.} By checking against the already used positions, the algorithm guarantees that no row is used by two queens.

The \texttt{step} method chooses two random indices and then switches the values (i.e. the queen's row) of the respective columns. Since the search algorithm is probabilistic anyways, it is not a problem if both indices are the same and the new constellation ends up the same as the old one.

The \texttt{combine} method applies \emph{cross-over} recombination: A new solution candidate is generated by assigning the first part of the columns their respective values from the first ``parent'' and filling the remaining columns with the remaining values from the second one. The split between the two parts happens at a randomly generated index. This kind of recombination resembles the way chromosomes are chosen form both parents in biological reproduction.

\section{Performing the search}

The domain discussed in the previous section \ref{sec:app:queens:domain} is used to register the variable \texttt{queens}, which saves a constellation of queens on the chess board. For this problem, there only is one constraint (\texttt{"check"}), which counts the number of queens who check one another.\footnote{By the nature of the problem, this number is always even.} Since it only needs to test for diagonal checks, for a selected queen, it iterates over every other queen and checks if its horizontal distance to the first queen equals the vertical distance. The comparison function passed to the constraint environment then chooses the solution candidate with the lesser number of checks.

\begin{lstlisting}[language=lua, caption={The script \texttt{queens} (part 2, continued from listing \ref{lst:queens1})}, label=lst:queens2, name=lst:queens]
board = constraints.environment:new(function (left, right)
    return left.check > right.check
end)
board:register("queens", queensdomain)
board:register("_", domains.just(0.8))

board:constrain("check", function()
    local checks = 0
    for q,queen in pairs(queens) do
        for o,other in pairs(queens) do
            if not (q == o) then
                if math.abs(topology[q] - topology[o]) == math.abs(queen - other) then
                    checks = checks + 1
                end
            end
        end
    end
    return checks
end)

mode = arg[1] or "both"

if mode == "evolution" or mode == "both" then
    print("+---+ evolutionary algorithm +-------------------+")
    constellations = evolution.population:new(board)
    constellations:seed(100)
    serialize.print(constellations:rank(constellations:best()), constellations:best():gettraits())
    
    constellations:age(100, 0.05, 1.0)
    serialize.print(constellations:rank(constellations:best()), constellations:best():gettraits())
end

if mode == "sa" or mode == "both" then
    print("+---+ simulated annealing +----------------------+")
    constellation = sa.process:new(board)
    constellation:start("check", function (t) return 1000-t end)
    serialize.print(constellation:rank(constellation:best()), constellation:best())
    
    constellation:anneal(1000)
    serialize.print(constellation:rank(constellation:best()), constellation:best())
end\end{lstlisting}

When running the program, it accepts a parameter telling it which search method to use. By default, it executes both a genetic algorithm and a simulated annealing process for testing purposes. See \ref{sec:app:queensperformance} for a comparison of both modules when solving the 8 queens problem.

The configuration of the genetic algorithm follows the parameters given in table 2.1 of \cite{EibenSmith2007} as closely as possible.\footnote{Since the module \texttt{evolution} only supports \emph{tournament selection} (cf. section 3.7.4 of \cite{EibenSmith2007}), both parent selection and survivor selection need to work differently.} It runs for 100 generations with a population of 100.

For simulated annealing, the \texttt{"check"} constraint can be used as an energy constraint, since it is an adequate heuristic of the quality of a candidate solution. The annealing process in 10000 steps features about as many new candidate evaluations as the genetic algorithm above. Because there is a usable heuristic, the simulated annealing search generally is better suited for this problem than the genetic algorithm.


\chapter{Test Programs}

\section{Performance measurement for \texttt{oo}'s objects}
\label{sec:app:ooperformance}

The measurements listed in table \ref{tab:performance} in section \ref{sec:oo:performance} were made using the program from listing \ref{lst:ooperformance} interpreted by Lua 5.1.5 under OS X 10.8.1. 

\begin{lstlisting}[language=lua, caption={Test program for memory measurement of different object types}, label=lst:ooperformance, name=lst:ooperformance]
require "oo"

oo.default(arg[1] or "lol")
base = arg[2] or 9000
mult = arg[3] or 10

all = {}
copies = {}
all[0] = oo.object
for i = 1,base do
	all[i] = all[i-1]:intend{
		prop = 5,
		["test"..tostring(i)] = oo.public (function (self)
			print("do some stuff")
		end)
	}
	copies[i] = {}
	for j = 1,mult do
		copies[i][j] = all[i]:clone()
	end
end

if arg[4] == "wait" then os.execute("sleep 10") end
if arg[4] == "inf"  then
	while true do
		--infinite loop, please kill process manually
	end
end
\end{lstlisting}
	
The first argument to the program sets the object type to be used for the test. The second and third arguments define the number of objects with which multiple arrays are filled.\footnote{It is necessary to keep a reference to all the created objects since Lua features automatic garbage collection.} The fourth argument allows to pause the program's execution for a while (\texttt{"wait"}) or indefinitely (\texttt{"inf"}) in order to ease using looking up the used memory manually.

\section{Self-adaptation example for \texttt{evolution}}
\label{sec:app:adaptation}

To provide the data of the example run of evolution strategies presented in table \ref{tab:esrun} in section \ref{sec:evolution:adaptation}, a script was used that repeatedly ages a population by one generation and then prints the best individual to the standard output. Along with the relevant traits of the best individual, it also computes the average traits of the whole population, using the handy \texttt{average} method of \texttt{evolution.population}, which only requires to be given the name of the respective (numerical) solution variable. The script in listing \ref{lst:adaptationperformance} includes the constraints environment as already shown in listings \ref{lst:constraints1}, \ref{lst:sa} and \ref{lst:evolution}.

\begin{lstlisting}[language=lua, caption={Test program for self-adaptation of the mutation rate in the module \texttt{evolution}}, label=lst:adaptationperformance, name=lst:adaptationperformance]
require "evolution"
require "domains"
require "constraints"
require "serialize"

x = constraints.environment:new(function(left, right)
	if left.bigdifference > right.bigdifference then
		return true
	elseif left.bigdifference < right.bigdifference then
		return false
	else
		return left.highc < right.highc
	end
end)
x:constrain("bigdifference", function() return math.abs(5-b) end)
x:constrain("highc", function() return c end)
x:register("b", domains.natural(2, 150))
x:register("c", domains.float(1, 19))

x:register("_", domains.float())
x:live()

e = evolution.population:new(x)
e:seed(10)


n = 1
print(n, b, c, _G["_"], e:average("b"), e:average("c"), e:average("_"))
for i=0,20 do
    e:age(1, 1, 0.9)
    n = n + 1
    print(n, b, c, _G["_"], e:average("b"), e:average("c"), e:average("_"))
end
\end{lstlisting}


\section{Comparison of \texttt{queens}' search algorithms}
\label{sec:app:queensperformance}

To compare the effectiveness of the search methods provided by the modules \texttt{sa} and \texttt{evolution}, the script \texttt{queenstat} uses the data structures defined in the script \texttt{queens} (see appendix \ref{chap:app:queens}) and executes each search algorithm 100 times. The data is collected in tables and printed out in a format comprehensible by most graph plotters.\footnote{The program also outputs a \LaTeX-style table for the purpose of including the data in table \ref{tab:queenstat}.}

\begin{lstlisting}[language=lua, caption={Test program \texttt{queenstat} for collecting statistical data about the search methods used in \texttt{queens} (see appendix \ref{chap:app:queens})}, label=lst:queenstat, name=lst:queenstat]
arg[1] = "none"
require "queens"

n = 100
results = {}
results["sa"] = {}
results["evolution"] = {}
avgs = {}
avgs["sa"] = {}
avgs["evolution"] = {}
for i=1,n do
    results["sa"][i] = {}
    constellation = sa.process:new(board)
    constellation:start("check", function (t) return 1000-t end)
    constellation:anneal(100)
    for c=1,100 do
        constellation:anneal(100)
        results["sa"][i][c] = constellation:rank(constellation:best())["check"]
    end
end
for i=1,n do
    results["evolution"][i] = {}
    constellations = evolution.population:new(board)
    constellations:seed(100)
    for c=1,100 do
        constellations:age(1, 0.05, 1.0)
        results["evolution"][i][c] = constellations:rank(constellations:best())["check"]
    end
end

table = "\\# & sa & ev.\\\\\n"
sagraph = ""
evgraph = ""
for c=1,100 do
    local sum = 0
    for i=1,n do
        sum = sum + results["sa"][i][c]
    end
    avgs["sa"][c] = sum / n
    sum = 0
    for i=1,n do
        sum = sum + results["evolution"][i][c]
    end
    avgs["evolution"][c] = sum / n
    table = table..tostring(c).." & "..tostring(avgs["sa"][c]).." & "..tostring(avgs["evolution"][c]).."\\\\\n"
    sagraph = sagraph..tostring(c).." "..tostring(avgs["sa"][c]).."\n"
    evgraph = evgraph..tostring(c).." "..tostring(avgs["evolution"][c]).."\n"
end

print(table)
print()
print(sagraph)
print()
print(evgraph)
\end{lstlisting}

\begin{table}[h]
\caption{Statistical data collected by the script of listing \ref{lst:queenstat}}
\begin{minipage}[t]{0.24\linewidth}
\begin{tabular}{|r|r|r|}
\hline
\# & sa & ev.\\
\hline
\hline
1 & 1.56 & 1.76\\
2 & 1.12 & 1.76\\
3 & 0.96 & 1.76\\
4 & 0.82 & 1.76\\
5 & 0.64 & 1.76\\
6 & 0.5 & 1.76\\
7 & 0.4 & 1.76\\
8 & 0.34 & 1.74\\
9 & 0.24 & 1.72\\
10 & 0.22 & 1.7\\
11 & 0.22 & 1.68\\
12 & 0.18 & 1.68\\
13 & 0.16 & 1.64\\
14 & 0.14 & 1.62\\
15 & 0.14 & 1.56\\
16 & 0.12 & 1.56\\
17 & 0.12 & 1.54\\
18 & 0.12 & 1.52\\
19 & 0.1 & 1.5\\
20 & 0.08 & 1.46\\
21 & 0.08 & 1.42\\
22 & 0.06 & 1.42\\
23 & 0.06 & 1.42\\
24 & 0.02 & 1.42\\
25 & 0.02 & 1.4\\
\hline
\end{tabular}
\end{minipage}
\begin{minipage}[t]{0.24\linewidth}
\begin{tabular}{|r|r|r|}
\hline
\# & sa & ev.\\
\hline
\hline
26 & 0.02 & 1.4\\
27 & 0.02 & 1.36\\
28 & 0.02 & 1.34\\
29 & 0.02 & 1.32\\
30 & 0.02 & 1.3\\
31 & 0.02 & 1.3\\
32 & 0 & 1.3\\
33 & 0 & 1.28\\
34 & 0 & 1.26\\
35 & 0 & 1.26\\
36 & 0 & 1.26\\
37 & 0 & 1.24\\
38 & 0 & 1.24\\
39 & 0 & 1.24\\
40 & 0 & 1.22\\
41 & 0 & 1.22\\
42 & 0 & 1.22\\
43 & 0 & 1.2\\
44 & 0 & 1.2\\
45 & 0 & 1.2\\
46 & 0 & 1.2\\
47 & 0 & 1.18\\
48 & 0 & 1.18\\
49 & 0 & 1.18\\
50 & 0 & 1.18\\
\hline
\end{tabular}
\end{minipage}
\begin{minipage}[t]{0.24\linewidth}
\begin{tabular}{|r|r|r|}
\hline
\# & sa & ev.\\
\hline
\hline
51 & 0 & 1.18\\
52 & 0 & 1.18\\
53 & 0 & 1.18\\
54 & 0 & 1.18\\
55 & 0 & 1.16\\
56 & 0 & 1.16\\
57 & 0 & 1.16\\
58 & 0 & 1.16\\
59 & 0 & 1.16\\
60 & 0 & 1.16\\
61 & 0 & 1.16\\
62 & 0 & 1.16\\
63 & 0 & 1.16\\
64 & 0 & 1.14\\
65 & 0 & 1.12\\
66 & 0 & 1.12\\
67 & 0 & 1.12\\
68 & 0 & 1.12\\
69 & 0 & 1.12\\
70 & 0 & 1.12\\
71 & 0 & 1.12\\
72 & 0 & 1.12\\
73 & 0 & 1.12\\
74 & 0 & 1.12\\
75 & 0 & 1.12\\
\hline
\end{tabular}
\end{minipage}
\begin{minipage}[t]{0.24\linewidth}
\begin{tabular}{|r|r|r|}
\hline
\# & sa & ev.\\
\hline
\hline
76 & 0 & 1.1\\
77 & 0 & 1.1\\
78 & 0 & 1.1\\
79 & 0 & 1.1\\
80 & 0 & 1.1\\
81 & 0 & 1.1\\
82 & 0 & 1.1\\
83 & 0 & 1.1\\
84 & 0 & 1.1\\
85 & 0 & 1.1\\
86 & 0 & 1.1\\
87 & 0 & 1.1\\
88 & 0 & 1.1\\
89 & 0 & 1.08\\
90 & 0 & 1.08\\
91 & 0 & 1.06\\
92 & 0 & 1.06\\
93 & 0 & 1.04\\
94 & 0 & 1.04\\
95 & 0 & 1.04\\
96 & 0 & 1.02\\
97 & 0 & 1.02\\
98 & 0 & 1.02\\
99 & 0 & 1\\
100 & 0 & 1\\
\hline
\end{tabular}
\end{minipage}
\label{tab:queenstat}
\end{table}

\begin{figure}[H]
\includegraphics[scale=0.54]{graphics/comparison}
\caption{Comparison between \texttt{sa} (below) and \texttt{evolution} (above)}
\label{fig:queenstat}
\end{figure}

The table \ref{tab:queenstat} and the figure \ref{fig:queenstat} show the statistical data gathered by one run of \texttt{queenstat} (see listing \ref{lst:queenstat}). The field ``\#'' denotes the current cycle within the search: For simulated annealing (sa), one cycle consists of 100 anneal steps, while for the genetic algorithm (ev.), one cycle equals one new generation. Thus, for both search algorithms, one cycle contains \emph{roughly} the same amount of newly generated solution candidates.\footnote{Also note that the the simulated annealing algorithm already executes 100 annealing steps before the first cycle in order to make up for the genetic algorithm starting with a sample of 100 individuals.} The other fields (``sa'' and ``ev.'') show the \emph{average} best rank achieved at the respective cycle in 100 independent executions of the algorithm. Note that the sampling done before the first cycle is already enough to warrant ``quite good'' results, with the average of checks never exceeding 2.

As becomes obvious from analyzing the data above, the simulated annealing algorithm is much more suited for this kind of problem, as it managed to solve it after at most 32 cycles in all of the 100 example runs. Of the genetic algorithm, only about half of the runs reached the optimal result after 100 cycles. However, as discussed in section \ref{sec:evolution:basics}, this was to be expected because the module \texttt{sa} is able to abuse the readily available heuristic while the module \texttt{evolution} is programmed not to in order to achieve a bigger flexibility when facing ill-specified problems, like they often appear in real-life adaptation processes.

\chapter{Documentation}

\section{Tools}

\begin{table}[H]
\begin{tabular}{|p{5cm}|p{8cm}|}
\hline
\multicolumn{2}{|c|}{Module \texttt{serialize}}\\
\hline
\hline
\texttt{command(name, param)} & serializes a call of given \texttt{command} with given \texttt{param}\\
\hline
\texttt{data(param)} & serializes given \texttt{param} to a call returning it\\
\hline
\texttt{print(...)} & serialize, then call Lua's \texttt{print}\\
\hline
\end{tabular}
\end{table}

\begin{table}[H]
\begin{tabular}{|p{5cm}|p{8cm}|}
\hline
\multicolumn{2}{|c|}{Module \texttt{typetools}}\\
\hline
\hline
\texttt{deepcopy(thing, cache)} & return deep copy of \texttt{thing} (\texttt{cache} is an accumulator, leave it out when calling from the outside)\\
\hline
\texttt{pairs(table)} & replacement of Lua's \texttt{pairs} which respects metamethod \texttt{\_\_pairs}\\
\hline
\texttt{proxy(target)} & return shallow copy of \texttt{target} that generates deep copies of its parts on demand, using metatables\\
\hline
\texttt{shallowcopy(thing)} & return a shallow copy of \texttt{thing}\\
\hline
\texttt{update(table, update)} & return copy of \texttt{table}, but overwrite with values from \texttt{update}\\
\hline
\end{tabular}
\end{table}

\begin{table}[H]
\begin{tabular}{|p{5cm}|p{8cm}|}
\hline
\multicolumn{2}{|c|}{Module \texttt{nd}}\\
\hline
\hline
\texttt{orelse(...)} & only execute first given function to return \texttt{true}\\
\hline
\texttt{whatif(call, globalsetting, localsetting)} & execute \texttt{call} in specified environment\\
\hline
\end{tabular}
\end{table}


\section{Message communication}

\begin{table}[H]
\begin{tabular}{|p{5cm}|p{8cm}|}
\hline
\multicolumn{2}{|c|}{Module \texttt{shared}}\\
\hline
\hline
\texttt{answer()} & answer longest pending message \\
\hline
\texttt{get(server, query)} & send get request message \\
\hline
\texttt{init(onsite, srv, clt)} & start messaging with given \texttt{onsite} commands with given server port and client port for network communication\\
\hline
\texttt{put(server, fact)} & send put request message \\
\hline
\texttt{qry(server, query)} & send qry request message \\
\hline
\texttt{term()} & terminate message communication \\
\hline
\texttt{stop()} & block the current coroutine \\
\hline
\end{tabular}
\end{table}

\begin{table}[H]
\begin{tabular}{|p{5cm}|p{8cm}|}
\hline
\multicolumn{2}{|c|}{Module \texttt{distributed}}\\
\hline
\hline
\texttt{answer()} & answer longest pending message \\
\hline
\texttt{get(server, query)} & send get request message \\
\hline
\texttt{init(onsite, me, qs)} & start messaging with given \texttt{onsite} commands, under given component \texttt{name} with given shared message queue (\texttt{qs})\\
\hline
\texttt{put(server, fact)} & send put request message \\
\hline
\texttt{qry(server, query)} & send qry request message \\
\hline
\texttt{term()} & terminate message communication \\
\hline
\texttt{stop()} & block the current coroutine \\
\hline
\end{tabular}
\end{table}


\section{Object Orientation}

\begin{table}[H]
\begin{tabular}{|p{5cm}|p{8cm}|}
\hline
\multicolumn{2}{|c|}{Module \texttt{oo}}\\
\hline
\hline
\texttt{default(type)} & set \texttt{type} of \texttt{oo.origin} \\
\hline
\texttt{dynamic(property)} & tag \texttt{property} as dynamic \\
\hline
\texttt{instantiate(...)} & generate instantiation method for enumerated properties \\
\hline
\texttt{getter(...)} & generate getter method for enumerated properties \\
\hline
\texttt{public(method)} & tag \texttt{method} as public \\
\hline
\texttt{object} & the prototype of all objects \\
\hline
\texttt{object:clone()} & \emph{clone the object} \\
\hline
\texttt{object:intend(intension)} & \emph{generate a new object by adding \texttt{intension}}\\
\hline
\texttt{object:new()} & \emph{generate a new object by adding nothing} \\
\hline
\texttt{setter(...)} & generate setter method for enumerated properties \\
\hline
\end{tabular}
\end{table}

\begin{table}[H]
\begin{tabular}{|p{5cm}|p{8cm}|}
\hline
\multicolumn{2}{|c|}{Module \texttt{ootypes}}\\
\hline
\hline
\texttt{lol} & the \texttt{"lol"} object type \\
\hline
\texttt{noo} & the \texttt{"noo"} object type \\
\hline
\texttt{tab} & the \texttt{"tab"} object type \\
\hline
\end{tabular}
\end{table}


\section{Probabilistic search}

\begin{table}[H]
\begin{tabular}{|p{5cm}|p{8cm}|}
\hline
\multicolumn{2}{|c|}{Module \texttt{constraints}}\\
\hline
\hline
\texttt{environment} & the prototype of the constraint environment \\
\hline
\texttt{environment:compare(left, right)} & \emph{compare \texttt{left} and \texttt{right} argument using the registered comparison function}\\
\hline
\texttt{environment:constrain( name, constraint)} & \emph{register \texttt{constraint} under given \texttt{name}}\\
\hline
\texttt{environment:current()} & \emph{retrieve currently best solution found}\\
\hline
\texttt{environment:getvars()} & \emph{retrieve the registered variables}\\
\hline
\texttt{environment:live(living)} & \emph{set live mode (true/false) to \texttt{living}}\\
\hline
\texttt{environment:max(...)} & \emph{call \texttt{test} method with given args}\\
\hline
\texttt{environment:new()} & \emph{generate a new environment}\\
\hline
\texttt{environment:register( varname, default)} & \emph{register global \texttt{varname} with \texttt{default} metadata}\\
\hline
\texttt{environment:test( settings)} & \emph{retrieve best index and best ranks of all given \texttt{settings}}\\
\hline
\texttt{environment:try(setting)} & \emph{retrieve ranks of \texttt{setting}}\\
\hline
\texttt{environment:update( values)} & \emph{set \texttt{values} as currently best solution}\\
\hline
\end{tabular}
\end{table}

\begin{table}[H]
\begin{tabular}{|p{5cm}|p{8cm}|}
\hline
\multicolumn{2}{|c|}{Module \texttt{domains}}\\
\hline
\hline
\texttt{float(min, max, deviation)} & generates a floating point number domain between \texttt{min} and \texttt{max} with given \texttt{deviation} \\
\hline
\texttt{just(value)} & generates a fixed domain containing just the given \texttt{value} \\
\hline
\texttt{natural(min, max, stepsize)} & generates an integer domain between \texttt{min} and \texttt{max} with given \texttt{stepsize} \\
\hline
\end{tabular}
\end{table}

\begin{table}[H]
\begin{tabular}{|p{5cm}|p{8cm}|}
\hline
\multicolumn{2}{|c|}{Module \texttt{sa}}\\
\hline
\hline
\texttt{process} & the prototype of the simulated annealing process \\
\hline
\texttt{process:anneal(n, t)} & \emph{make \texttt{n} annealing steps (\texttt{t} exists for technical reasons, leave it out)}\\
\hline
\texttt{process:annealto(t)} & \emph{make one annealing step to temperature \texttt{t}}\\
\hline
\texttt{process:best()} & \emph{return currently best solution found}\\
\hline
\texttt{process:compare(left, right)} & \emph{compare \texttt{left} and \texttt{right} argument using the environment's comparison function}\\
\hline
\texttt{process:new(environment, domains)} & \emph{generate a new simulated annealing process based on \texttt{environment} but overwrite with \texttt{domains} (optional)}\\
\hline
\texttt{process:rank(state)} & \emph{retrieve rank of given \texttt{state}}\\
\hline
\texttt{process:start(energy, temperature)} & \emph{start simulated annealing on the given \texttt{energy} constraint with given \texttt{temperature} function}\\
\hline
\end{tabular}
\end{table}

\begin{table}[H]
\begin{tabular}{|p{5cm}|p{8cm}|}
\hline
\multicolumn{2}{|c|}{Module \texttt{evolution}}\\
\hline
\hline
\texttt{individual} & the prototype of the individuals of an evolutionary algorithm\\
\hline
\texttt{individual:gettraits()} & \emph{return traits}\\
\hline
\texttt{individual:mutate()} & \emph{cause individual to undergo mutation}\\
\hline
\texttt{individual:new(domains, traits)} & \emph{create new individual based on given \texttt{domains} and set \texttt{traits} (optional)}\\
\hline
\texttt{individual:recombine( other)} & \emph{generate new individual by recombining this one with \texttt{other}}\\
\hline
\texttt{population} & the prototype of the population of an evolutionary algorithm\\
\hline
\texttt{population:age(n, meetrate, birthrate)} & \emph{age population for \texttt{n} generations with given \texttt{meetrate} and \texttt{birthrate}}\\
\hline
\texttt{population:all(trait)} & \emph{collect all values of \texttt{trait} occuring in the population in a table}\\
\hline
\texttt{population:average(trait)} & \emph{compute the average of numerical \texttt{trait} across the population}\\
\hline
\texttt{population:best()} & \emph{return currently best individual found}\\
\hline
\texttt{population:compare(left, right)} & \emph{compare \texttt{left} and \texttt{right} individuals using the environment's comparison function}\\
\hline
\texttt{population:grow(n)} & \emph{grow by \texttt{n} individuals through reproduction}\\
\hline
\texttt{population:new( environment, domains)} & \emph{create new population based on \texttt{environment} but overwrite with \texttt{domains} (optional)}\\
\hline
\texttt{population:rank( individual)} & \emph{retrieve rank of given \texttt{individual}}\\
\hline
\texttt{population:seed(n)} & \emph{create \texttt{n} new individuals as members}\\
\hline
\texttt{population:survive( meetrate, birthrate)} & \emph{apply selection with given \texttt{meetrate} and \texttt{birthrate}} \\
\hline
\end{tabular}
\end{table}



\begin{comment}
\chapter{A quick guide to Lua's oddities}
\end{comment}

\end{appendices}